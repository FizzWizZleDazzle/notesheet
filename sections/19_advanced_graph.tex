\section{Advanced Graph Algorithms}
% ============================================================

\subsection{Strongly Connected Components (Tarjan's)}
\begin{desc}
In a directed graph, an SCC is a maximal set of nodes where every node is
reachable from every other node.\\
\textbf{Tarjan's algorithm}: single DFS with a stack, tracking discovery time
(\texttt{disc}) and lowest reachable ancestor (\texttt{low}).
When \texttt{low[u] == disc[u]}, $u$ is the root of an SCC.\\
\textbf{Use for}: directed graph analysis, finding cycles, condensation graph.
\end{desc}
\complexity{$O(V + E)$}{$O(V)$}
\begin{lstlisting}
static int[] disc, low, sccId;
static boolean[] onStack;
static Deque<Integer> stack;
static int time = 0, sccCount = 0;

static void tarjanSCC(int u, List<List<Integer>> adj) {
  disc[u] = low[u] = time++;
  stack.push(u);
  onStack[u] = true;

  for (int v : adj.get(u)) {
    if (disc[v] == -1) {
      tarjanSCC(v, adj);
      low[u] = Math.min(low[u], low[v]);
    } else if (onStack[v]) {
      low[u] = Math.min(low[u], disc[v]);
    }
  }

  // If u is root of SCC, pop the SCC
  if (low[u] == disc[u]) {
    while (true) {
      int v = stack.pop();
      onStack[v] = false;
      sccId[v] = sccCount;
      if (v == u) break;
    }
    sccCount++;
  }
}

// Usage
static void findSCCs(int n, List<List<Integer>> adj) {
  disc = new int[n];
  low = new int[n];
  sccId = new int[n];
  onStack = new boolean[n];
  stack = new ArrayDeque<>();
  Arrays.fill(disc, -1);
  time = 0; sccCount = 0;

  for (int i = 0; i < n; i++)
    if (disc[i] == -1) tarjanSCC(i, adj);
}
\end{lstlisting}

\subsection{Bridges and Articulation Points (Tarjan's)}
\begin{desc}
\textbf{Bridge}: an edge whose removal disconnects the graph.\\
\textbf{Articulation point}: a vertex whose removal disconnects the graph.\\
Both found using a single DFS with \texttt{disc} (discovery time) and \texttt{low}
(lowest reachable ancestor).\\
\textbf{Bridge}: edge $(u,v)$ is a bridge if $\text{low}[v] > \text{disc}[u]$.\\
\textbf{Articulation point}: $u$ is one if it's the root with $\ge 2$ children,
or non-root with a child $v$ where $\text{low}[v] \ge \text{disc}[u]$.
\end{desc}
\complexity{$O(V + E)$}{$O(V)$}
\begin{lstlisting}
static int[] disc, low;
static boolean[] isAP;
static List<int[]> bridges;
static int time = 0;

static void dfs(int u, int p, List<List<Integer>> adj) {
  disc[u] = low[u] = time++;
  int children = 0;

  for (int v : adj.get(u)) {
    if (disc[v] == -1) {
      children++;
      dfs(v, u, adj);
      low[u] = Math.min(low[u], low[v]);

      // Check for bridge
      if (low[v] > disc[u])
        bridges.add(new int[]{u, v});

      // Check for articulation point (non-root)
      if (p != -1 && low[v] >= disc[u])
        isAP[u] = true;
    } else if (v != p) {
      // Back edge
      low[u] = Math.min(low[u], disc[v]);
    }
  }

  // Check if root is articulation point
  if (p == -1 && children > 1)
    isAP[u] = true;
}

// Usage
static void findBridgesAndAPs(int n,
    List<List<Integer>> adj) {
  disc = new int[n];
  low = new int[n];
  isAP = new boolean[n];
  bridges = new ArrayList<>();
  Arrays.fill(disc, -1);
  time = 0;

  for (int i = 0; i < n; i++)
    if (disc[i] == -1) dfs(i, -1, adj);
}
\end{lstlisting}

\subsection{Eulerian Path / Circuit (Hierholzer's)}
\begin{desc}
An \textbf{Eulerian path} visits every edge exactly once; an \textbf{Eulerian circuit}
also returns to the start.\\
\textbf{Undirected}: circuit exists iff all degrees even; path exists iff exactly
2 odd-degree nodes (start and end at them).\\
\textbf{Directed}: circuit iff $\text{in}=\text{out}$ for all nodes; path iff one node
has $\text{out}-\text{in}=1$ (start) and one has $\text{in}-\text{out}=1$ (end).\\
Hierholzer's runs in $O(E)$ using a stack.
\end{desc}
\complexity{$O(V + E)$}{$O(V + E)$}
\begin{lstlisting}
// Undirected Eulerian path/circuit (use edge indices to avoid re-traversal)
// adj[u] = list of {v, edgeIndex}
static List<Integer> eulerPath(int n, int m,
    List<int[]>[] adj, int start) {
  var used = new boolean[m];  // edge used?
  var it = new int[n]; // iterator per node
  var stk = new ArrayDeque<Integer>();
  var path = new ArrayList<Integer>();
  stk.push(start);
  while (!stk.isEmpty()) {
    int u = stk.peek();
    boolean found = false;
    while (it[u] < adj[u].size()) {
      var e = adj[u].get(it[u]++);
      if (!used[e[1]]) {
        used[e[1]] = true;
        stk.push(e[0]);
        found = true; break;
      }
    }
    if (!found) { path.add(u); stk.pop(); }
  }
  Collections.reverse(path);
  return path; // size == m+1 if Eulerian path exists
}
\end{lstlisting}

\subsection{Maximum Flow (Edmonds-Karp)}
\begin{desc}
Finds the maximum flow from source $s$ to sink $t$ using BFS to find
\textbf{shortest augmenting paths} (Edmonds-Karp variant of Ford-Fulkerson).\\
Maintains a \textbf{residual graph}: each edge $(u,v,c)$ adds forward cap $c$
and backward cap $0$. Augmenting pushes flow and updates residuals.\\
\textbf{Min-cut} = max-flow (max-flow min-cut theorem).
\end{desc}
\complexity{$O(VE^2)$}{$O(V^2)$}
\begin{lstlisting}
static int N; // number of nodes
static int[][] cap; // residual capacity matrix

static int bfs(int s, int t, int[] parent) {
  Arrays.fill(parent, -1);
  parent[s] = s;
  Queue<int[]> q = new ArrayDeque<>();
  q.offer(new int[]{s, Integer.MAX_VALUE});
  while (!q.isEmpty()) {
    var cur = q.poll();
    int u = cur[0], f = cur[1];
    for (int v = 0; v < N; v++) {
      if (parent[v] == -1 && cap[u][v] > 0) {
        parent[v] = u;
        int nf = Math.min(f, cap[u][v]);
        if (v == t) return nf;
        q.offer(new int[]{v, nf});
      }
    }
  }
  return 0;
}

static int maxFlow(int s, int t) {
  int flow = 0, pushed;
  var parent = new int[N];
  while ((pushed = bfs(s, t, parent)) > 0) {
    flow += pushed;
    int v = t;
    while (v != s) {
      int u = parent[v];
      cap[u][v] -= pushed;
      cap[v][u] += pushed;
      v = u;
    }
  }
  return flow;
}
// Init: cap = new int[N][N]; add edge (u,v,c):
//   cap[u][v] += c; (cap[v][u] += c for undirected)
\end{lstlisting}

\subsection{Bipartite Maximum Matching}
\begin{desc}
Find the largest set of edges in a bipartite graph such that each vertex
is covered by at most one edge.\\
Uses \textbf{augmenting paths} (Hopcroft-Karp idea simplified): for each left
vertex, try to find a free right vertex via DFS, possibly rerouting existing matches.\\
\textbf{Kőnig's theorem}: min vertex cover = max matching in bipartite graphs.
\end{desc}
\complexity{$O(V \cdot E)$}{$O(V + E)$}
\begin{lstlisting}
static int L, R; // left and right sizes
static List<Integer>[] adjL; // adjL[u] = right neighbors
static int[] matchL, matchR;
static boolean[] visited;

static boolean dfs(int u) {
  for (int v : adjL[u]) {
    if (!visited[v]) {
      visited[v] = true;
      if (matchR[v] == -1 || dfs(matchR[v])) {
        matchL[u] = v; matchR[v] = u;
        return true;
      }
    }
  }
  return false;
}

static int maxMatching() {
  matchL = new int[L]; matchR = new int[R];
  Arrays.fill(matchL, -1); Arrays.fill(matchR, -1);
  int res = 0;
  for (int u = 0; u < L; u++) {
    visited = new boolean[R];
    if (dfs(u)) res++;
  }
  return res;
}
// matchL[u] = right partner of u (-1 if unmatched)
// matchR[v] = left partner of v (-1 if unmatched)
\end{lstlisting}

\subsection{Cycle Detection}
\begin{desc}
Detect if a graph contains a cycle.\\
\textbf{Undirected}: simple DFS---if we visit a neighbor that's already visited
and it's not our parent, cycle exists.\\
\textbf{Directed}: 3-color DFS (white/gray/black)---gray nodes are currently
in the DFS stack. A back edge to a gray node means cycle.
\end{desc}
\complexity{$O(V + E)$}{$O(V)$}
\begin{lstlisting}
// Undirected graph cycle detection
static boolean hasCycleUndirected(int u, int p,
    List<List<Integer>> adj, boolean[] vis) {
  vis[u] = true;
  for (int v : adj.get(u)) {
    if (!vis[v]) {
      if (hasCycleUndirected(v, u, adj, vis))
        return true;
    } else if (v != p) {
      return true; // visited non-parent = cycle
    }
  }
  return false;
}

// Directed graph cycle detection (3-color)
static final int WHITE = 0, GRAY = 1, BLACK = 2;
static boolean hasCycleDirected(int u, int[] color,
    List<List<Integer>> adj) {
  color[u] = GRAY;
  for (int v : adj.get(u)) {
    if (color[v] == GRAY) return true; // back edge
    if (color[v] == WHITE &&
        hasCycleDirected(v, color, adj))
      return true;
  }
  color[u] = BLACK;
  return false;
}
\end{lstlisting}

\subsection{Bipartite Graph Check}
\begin{desc}
A graph is \textbf{bipartite} if its nodes can be colored with 2 colors such that
no edge connects same-colored nodes. Equivalent to having no odd-length cycles.\\
\textbf{Algorithm}: 2-color BFS or DFS. Start with color 0, alternate colors.
If a neighbor already has the same color, not bipartite.
\end{desc}
\complexity{$O(V + E)$}{$O(V)$}
\begin{lstlisting}
static boolean isBipartite(int n,
    List<List<Integer>> adj) {
  var color = new int[n];
  Arrays.fill(color, -1);

  for (int start = 0; start < n; start++) {
    if (color[start] != -1) continue;

    Queue<Integer> q = new ArrayDeque<>();
    q.offer(start);
    color[start] = 0;

    while (!q.isEmpty()) {
      int u = q.poll();
      for (int v : adj.get(u)) {
        if (color[v] == -1) {
          color[v] = 1 - color[u];
          q.offer(v);
        } else if (color[v] == color[u]) {
          return false; // same color = not bipartite
        }
      }
    }
  }
  return true;
}
\end{lstlisting}

% ============================================================
